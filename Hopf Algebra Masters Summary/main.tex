\documentclass[12pt,a4paper]{article}
\usepackage[english]{babel}
\usepackage[utf8x]{inputenc}
\usepackage[T1]{fontenc}
\usepackage{listings}
\usepackage{amsfonts}
\usepackage{amssymb}
\usepackage{amsmath,amsthm}
\usepackage{mathtools}
\usepackage{graphicx}
\usepackage{geometry}
\usepackage{layout}
\usepackage{tikz}
\usepackage{bbm}
\usepackage{tikz-cd}
\usepackage{cite}
\usetikzlibrary{positioning}
\usepackage{latexsym,amsxtra,amscd,ifthen, amsmath,color,url}
\newcommand{\comp}{\overline{\phantom{A}}}
\setlength{\tabcolsep}{12pt}

\newtheorem{theorem}{Theorem}[section]
\newtheorem{lemma}[theorem]{Lemma}
\newtheorem{claim}[theorem]{Claim}
\newtheorem{proposition}[theorem]{Proposition}
\newtheorem{corollary}[theorem]{Corollary}
\newtheorem{fact}[theorem]{Fact}
\newtheorem{example}[theorem]{Example}
\newtheorem{notation}[theorem]{Notation}
\newtheorem{observation}[theorem]{Observation}
\newtheorem{conjecture}[theorem]{Conjecture}
\newtheorem{remark}[theorem]{Remark}
\newtheorem{definition}[theorem]{Definition}
\newcommand\NN{\mathbb{N}} 
\newcommand\RR{\mathbb{R}}
\newcommand\CR{\mathcal{R}}
\newcommand\ZZ{\mathbb{Z}}
\newcommand\AL{\mathcal{A}}
\newcommand\MM{\mathcal{M}}
\newcommand\BB{\mathcal{B}}
\newcommand\BBB{\mathfrak{B}}
\newcommand\CC{\mathcal{C}}
\newcommand\CCC{\mathfrak{C}}
\newcommand\DD{\mathcal{D}}
\newcommand\EE{\mathbb{E}}
\newcommand\FF{\mathcal{F}}
\newcommand\GG{\mathcal{G}}
\newcommand\HH{\mathcal{H}}
\newcommand\JJ{\mathcal{J}}
\newcommand\PP{\mathcal{P}}
\newcommand\QQ{\mathbb{Q}}
\newcommand\CS{\mathcal{S}}
\newcommand\TT{\mathcal{T}}
\newcommand\BT{\mathbb{T}}
\newcommand\OO{\mathcal{O}}
\newcommand\UU{\mathcal{U}}
\newcommand\VV{\mathcal{V}}
\newcommand\PPT{\mathcal{P}^{*}}
\newcommand\CST{\Sigma^{*}}
\newcommand\RX{\mathsf{X}}
\newcommand\1{_{(1)}}
\newcommand\2{_{(2)}}

%\voffset=-1truein		% LaTeX has too much space at page top
%\addtolength{\textheight}{0.3truein}
%\addtolength{\textheight}{\topmargin}
%\addtolength{\topmargin}{-\topmargin}
%\textwidth  6.0in		% LaTeX article default 360pt=4.98''
%\oddsidemargin 0pt	% \oddsidemargin  .35in   % default is 21.0 pt
%\evensidemargin 0pt	% \evensidemargin .35in   % default is 59.0 pt
%\marginparwidth=0pt
% decrease margins on sides and top/bottom
\geometry{
bottom=25mm,
top=35mm,
left=20mm,
right=20mm
}
\voffset=-0.5truein
\title{Hopf Algebras Acting on Quantum Planes}
\date{}
\author{Brandon Mather}
\begin{document}%\layout
	%\MakeScribeTop

\maketitle

\section{Hopf Algebra Definitions}
\begin{definition}
A Hopf Algebra, $\left(H,\bigtriangledown,\eta,\triangle,\varepsilon,S\right)$, is a bialgebra $H$ over a field $\mathbb{C}$ with an antipode $S:H\to H$ where the bialgebra has product $\bigtriangledown:H\otimes H\to H$, unit $\eta:\mathbb{C}\to H$, coproduct $\triangle:H\to H\otimes H$, counit $\varepsilon:H\to\mathbb{C}$ such that the following diagrams commute
% https://tikzcd.yichuanshen.de/#N4Igdg9gJgpgziAXAbVABwnAlgFyxMJZABgBoBGAXVJADcBDAGwFcYkQBBAHS4jwFt4AAg4gAvqXSZc+QijLFqdJq3bdeA4er5ZBcEeMkgM2PASLlSimgxZtEnHjr0GJU07IsUlt1Q9FiSjBQAObwRKAAZgBOEPxIliA4EEhkynbsWFBOmvo8ONFY9GAhjMEQAO6EbiAxcQk0yUgATDSM9ABGMIwACtJmciBlkTggNir2IPmFxaXlVTm6wlmGUbHxiGlNiADM4xkO00UlZVCV1UZ1G61JKbv7flNcBcdzZ1XilGJAA
\[
\text{Associativity:}
\begin{tikzcd}
H\otimes H\otimes H \arrow[d, "id\otimes \bigtriangledown"] \arrow[r, "\bigtriangledown\otimes id"] & H\otimes H \arrow[d, "\bigtriangledown"] \\
H\otimes H \arrow[r, "\bigtriangledown"]                                                         & H                                    
\end{tikzcd}
\text{   \;\;\;\;Unit:}
% https://tikzcd.yichuanshen.de/#N4Igdg9gJgpgziAXAbVABwnAlgFyxMJZARgBoAGAXVJADcBDAGwFcYkQBBAHS4jwFt4AAg4gAvqXSZc+QinKli1Ok1bseAaw08+WQXBHjJIDNjwEiZAEzKGLNok5GpZ2USuLbqh5x0DhmhriyjBQAObwRKAAZgBOEPxIZCA4EEgKKvbqXDA49H56wlhQziBxCUgAzDSp6TR2ao7FBfo8ufQgNIz0AEYwjAAK0uZyILFYYQAWOKXliYjJtYgemY0gALydIN19g8NujuNTMxIx8fPVKWnL9d7sm6dl53VXSCsNPjw44-RgYYyhCAAd0IYkoYiAA
\begin{tikzcd}
                                                           & H\otimes H \arrow[dd, "\bigtriangledown"] &                                                           \\
\mathbb{C}\otimes H \arrow[ru, "\eta\otimes id"] \arrow[rd, "="'] &                                        & H\otimes \mathbb{C} \arrow[lu, "id\otimes\eta"'] \arrow[ld, "="] \\
                                                           & H                                      &                                                          
\end{tikzcd}
\]
\[
\text{Coassociativity:} % https://tikzcd.yichuanshen.de/#N4Igdg9gJgpgziAXAbVABwnAlgFyxMJZABgBpiBdUkANwEMAbAVxiRAAkQBfU9TXfIRQBGclVqMWbdgB0ZEPAFt4AAk48+2PASJlh4+s1aIOchVmVw13XiAxbBRUfuqGpJ2fKWrP5y9a5xGCgAc3giUAAzACcIRSQyEBwIJFEJIzY5HGisOjAQhlYNEBi4hOpkpAAmV0ljECycvIKi21L4xBqklMQAZlqMk0bc-MKQajgACyxInATi9tSKnv709waZbJGWs28rLChuCi4gA
\begin{tikzcd}
H \arrow[r, "\triangle"] \arrow[d, "\triangle"] & H\otimes H \arrow[d, "\triangle\otimes id"] \\
H\otimes H \arrow[r, "id\otimes\triangle"]               & H\otimes H\otimes H                        
\end{tikzcd}
\text{\;\;\;\; Counit:}
% https://tikzcd.yichuanshen.de/#N4Igdg9gJgpgziAXAbVABwnAlgFyxMJZABgBoBGAXVJADcBDAGwFcYkQAdDga264jwBbeAAIAEiAC+pdJlz5CKcqWLU6TVuwnTZ2PASIAmCmoYs2iEGP5D4XXlJkgMehUWWHTGi1ZtZhcOJSajBQAObwRKAAZgBOEIJIAMw0OBBIZOrm7FwMsTBo2IwEfgEiWFCOMfGJiMogaRk0ZpqWALwgNIz0AEYwjAAKcvqKILFYYQAWOFUgcQlI9Y2IxlmtIB06czXJqekrzd7sFaV2HHkFRQZdvf1DrgaW41MzW-O1S-spaz5cOOP0MBhRhsSSUSRAA
\begin{tikzcd}
             & H \arrow[ld, "="'] \arrow[rd, "="] \arrow[dd, "\triangle"]                         &             \\
\mathbb{C}\otimes H &                                                                                    & H\otimes\mathbb{C} \\
             & H\otimes H \arrow[lu, "\varepsilon\otimes id"] \arrow[ru, "id\otimes\varepsilon"'] &            
\end{tikzcd}
\]
\[
\text{Coproduct compatibility:}
% https://tikzcd.yichuanshen.de/#N4Igdg9gJgpgziAXAbVABwnAlgFyxMJZABgBpiBdUkANwEMAbAVxiRAAkAdTiPAW3gACdiAC+pdJlz5CKAIzkqtRizYjxk7HgJEATIur1mrRB268sAuMLESQGLTKJk5So6tNce-IV4tXhcx9rdTsHaR0UfVdDFRMzb0tfIKSQlID1JRgoAHN4IlAAMwAnCD4kMhAcCCQAZljjNm4cYqw6MByGGHT4Ztb2ztYNEBKyuupqpAAWBo8QLCgeuGa6JiXBBdsi0vLESsnEBWVG0z62jq6oCAB3QmHR3aOD-WO5s4GurZGd6YmaxBe7ni7wu2RuYCWIMGV1uYgooiAA
\begin{tikzcd}
H\otimes H \arrow[d, "\triangle\otimes\triangle"] \arrow[r, "\bigtriangledown"] & H \arrow[r, "\triangle"] & H\otimes H                                                                  \\
H\otimes H\otimes H\otimes H \arrow[rr, "id\otimes\tau\otimes id"]           &                          & H\otimes H\otimes H\otimes H \arrow[u, "\bigtriangledown\otimes\bigtriangledown"]
\end{tikzcd}
\]
\[
% https://tikzcd.yichuanshen.de/#N4Igdg9gJgpgziAXAbVABwnAlgFyxMJZABgBpiBdUkANwEMAbAVxiRAB12BrLkAX1LpMufIRQBGclVqMWbABL9BIDNjwEik8dPrNWiEPM4Q8AW3gACRX2kwoAc3hFQAMwBOEU0jIgcEJABM1LpyBpwwOHTGZvDhkUquHl6Ikr7+iEEyemycOG5YdGD2DKwCiZ7e1H5IqSH6HOwRdPwUfEA
\text{Unit compatibility:}\begin{tikzcd}
\mathbb{C} \arrow[rd, "\eta\otimes\eta"] \arrow[r, "\eta"] & H \arrow[d, "\triangle"] \\
                                                    & H\otimes H              
\end{tikzcd}
\text{\;\;\;Counit compatibility:}
% https://tikzcd.yichuanshen.de/#N4Igdg9gJgpgziAXAbVABwnAlgFyxMJZABgBpiBdUkANwEMAbAVxiRAAkQBfU9TXfIRQBGclVqMWbADrSA1nO68QGbHgJEyw8fWatEHWRDwBbeAAJOXcTCgBzeEVAAzAE4QTSAEzUcEJGQSejLSOK5YdGB2DLYQAO6EPC7unog+IH5IokFSBrL0rjBo2AwERqbw+XSFxVilhNQMdABGMAwACvzqQiDhdgAWOErJHgG+-ojZurkgVTUlGtZcQA
\begin{tikzcd}
H \arrow[r, "\varepsilon"]                                                         & \mathbb{C} \\
H\otimes H \arrow[u, "\bigtriangledown"] \arrow[ru, "\varepsilon\otimes\varepsilon"'] &    
\end{tikzcd}
\]
\[
\text{Antipode:}
% https://tikzcd.yichuanshen.de/#N4Igdg9gJgpgziAXAbVABwnAlgFyxMJZABgBoBGAXVJADcBDAGwFcYkQAJEAX1PU1z5CKchWp0mrdgB1pAazk8+IDNjwEiAJjE0GLNok5L+aoUTKbxeqYY6yIeALbwABF14nBGlNsu7JBpz2Tq7uyqpewiSkxFYB7HbSDljOcG7GKgLqUdqx-voJwSmhPOIwUADm8ESgAGYAThCOSGQgOBBIAKz5NiCyOPVY9GAVjGweIA1NXTTtSABsPYFYUEWpLgDKGVPNiIttHYjaEgWG-YPDo+UQAO6EEzsts4cAzEsy0gNDI2MgNIz0ABGMEYAAUsmZDIMKgALHDbRq7N4HJAAFnehg2a1cKwR00Q6JRRwxfU+Fx+1zueN2rTmiFEJ16sgY9RgaGwjA0D0RSAZdOO1kCshgOHopW4QA
\begin{tikzcd}
H\otimes H \arrow[rr, "id\otimes S"]                                      &                       & H\otimes H \arrow[d, "\bigtriangledown"] \\
H \arrow[u, "\triangle"] \arrow[d, "\triangle"'] \arrow[r, "\varepsilon"] & \mathbb{C} \arrow[r, "\eta"] & H                                     \\
H\otimes H \arrow[rr, "S\otimes id"]                                      &                       & H\otimes H \arrow[u, "\bigtriangledown"]
\end{tikzcd}
\]\\
\end{definition}

For the sake of brevity, we write in general that $\triangle(h)=\sum h\1\otimes h\2$, this is called Sweedler notation.\\

\begin{definition} A left action of a Hopf algebra on a vector space $V$ is a tuple $(\alpha, V)$ so that
$\alpha:H\otimes V\to V$ is a map satisfying the diagrams
\[
% https://tikzcd.yichuanshen.de/#N4Igdg9gJgpgziAXAbVABwnAlgFyxMJZABgBpiBdUkANwEMAbAVxiRAAkAdTiPAW3gACLj35CAaiAC+pdJlz5CKMgEYqtRizYjeWAXEGSZc7HgJEV5dfWatEHbrv2HpskBlOKLpNdRtb7I3UYKABzeCJQADMAJwg+JDIQHAgkACY-TTsQbhwYrDowUIYQiAB3MEcxAywoV2i4hMQklKRLDVs2Wqq9IW5GNAALOnqQWPi26lbEAGZMzvt+hiGR4zHG9KnU2fmAnM4B4ekKKSA
\begin{tikzcd}
H\otimes H\otimes V \arrow[r, "\bigtriangledown\otimes id"] \arrow[d, "id\otimes \alpha"] & H\otimes V \arrow[d, "\alpha"] \\
H\otimes V \arrow[r, "\alpha"]                                                         & V                             
\end{tikzcd}
% https://tikzcd.yichuanshen.de/#N4Igdg9gJgpgziAXAbVABwnAlgFyxMJZABgBpiBdUkANwEMAbAVxiRAB12BrLziPALbwABADUQAX1LpMufIRQBGclVqMWbABJ9BI8VJnY8BIssWr6zVohD7VMKAHN4RUADMAThAFIyIHBBIympWbJwwOHQ6WEJwwlhQktIgnt6+1AFIAEzUlho2ALwg1Ax0AEYwDAAKssYKIB5YjgAWOEnuXj6IwZmIOSH5HOyMaM10khQSQA
\begin{tikzcd}
\mathbb{C}\otimes V \arrow[r, "\eta\otimes id"] \arrow[rd, "="'] & H\otimes V \arrow[d, "\alpha"] \\
                                                          & V                             
\end{tikzcd}
\]
In this case, $V$ is called a left H-module.
\end{definition}

\begin{definition}
When an algebra $A$ is a left H-module with action $\alpha$, we call it a left H-module algebra if the following diagrams also commute
\[
% https://tikzcd.yichuanshen.de/#N4Igdg9gJgpgziAXAbVABwnAlgFyxMJZABgBpiBdUkANwEMAbAVxiRAAkAdTiPAW3gACAILdeWAXBEgAvqXSZc+QigCM5KrUYs2XHvyHDZ8kBmx4CRAEwbq9Zq0QgjcheeVEAzLa0O2o-QlDYzclSxQyVU17HSc9cUlBeIMpAITg11NFCxVkbyi7bUcOMRSRUqCpZMrpGU0YKABzeCJQADMAJwg+JDIQHAgkdV9YkG4cDqw6MEaGBogAd0JMzu6h6gGkGxHi7kY0AAs6EJBVnsRvfsHEbZjdzgmpmbmoRcJqBjoAIxgGAAVsh4nJNGgccCczr0NtcACyFPxOcaTaazGAVRJYKDooSYkAfb6-AHucIgEFgiFdc5wq5IACs8NGmOxcHGdCYzMEuJWlLp0KQlzubD2DEOdGZwtFeJAnx+-0BJLJ4LqMiAA
\begin{tikzcd}
H\otimes A\otimes A \arrow[r, "\bigtriangledown"] \arrow[d, "\triangle\otimes id\otimes id"'] & H\otimes A \arrow[r, "\alpha"] & A & A\otimes A \arrow[l, "\bigtriangledown"']                         \\
H\otimes H\otimes A\otimes A \arrow[rrr, "id\otimes\tau\otimes id"]                        &                                &   & H\otimes A\otimes H\otimes A \arrow[u, "\alpha\otimes\alpha"']
\end{tikzcd}
% https://tikzcd.yichuanshen.de/#N4Igdg9gJgpgziAXAbVABwnAlgFyxMJZABgBpiBdUkANwEMAbAVxiRAAkQBfU9TXfIRQBGclVqMWbADrSA1nO68QGbHgJEATGOr1mrRCACCSvmsFEyw8XqmH2siHgC28AAQmu4mFADm8IlAAMwAnCGckMhAcCCRRCX0ZaXoQmDRsBg0eYLCIxHiYpG0EuxBZGBw6UxBQ8MjqQsQAZl1JAzLpCqrsmtykFujYxGLbdtlGNAALboouIA
\begin{tikzcd}
H \arrow[r, "\varepsilon"] \arrow[d, "\eta"] & \mathbb{C} \arrow[r, "\eta"] & A \\
H\otimes A \arrow[rru, "\alpha"]             &                       &  
\end{tikzcd}
\]
\end{definition}

\begin{definition}
Let $Q=(q_{ij})$ be an $n\times n$ matrix of roots of unity where $q_{ii}=1=q_{ij}q_{ji}$.
\\A quantum polynomial ring is $\mathbb{C}_Q[v_1,\ldots,v_n]=\mathbb{C}\langle v_1,\ldots,v_n\vert v_jv_i=q_{ij}v_iv_j\rangle$.
\end{definition}

\begin{definition}
A quantum group is a Hopf algebra $H$ with a bijective antipode and an element $R\in H\otimes H$ satisfying 
\begin{enumerate}
    \item $R\left(\sum h\1\otimes h\2\right)R^{-1}=\sum h\2\otimes h\1$
    \item $\triangle\otimes id(R)=R_{1,3}R_{2,3}$
    \item $id\otimes\triangle(R)=R_{1,3}R_{1,2}$
\end{enumerate}
where, writing $R=\sum R\1\otimes R\2$, then $R_{1,2}=\sum R\1\otimes R\2\otimes 1$, $R_{1,3}=\sum R\1\otimes 1\otimes R\2$, and $R_{2,3}=\sum 1\otimes R\1\otimes R\2$.
\end{definition}

\section{Examples}

\begin{example}
    In his seminal book, "Hopf Algebras", Sweedler defined a $4$-dimensional non-commutative, non-cocommutative Hopf algebra
    \[
    H_4=\langle g,x\;\vert\; g^2=1, x^2=0,gx=-xg\rangle
    \]
    with operations
    \[
    \triangle{g}=g\otimes g,\;\; \triangle{x}=x\otimes 1+g\otimes x,\;\;\varepsilon(g)=1,\;\;\varepsilon(x)=0,
    \]
    \[
    S(g)=g^{-1}\;\;S(x)=-xg^{-1}.
    \]
   $\mathbb{C}_{-1}[v_1,v_2]$ is an $H_4$-module algebra via $g\mapsto \begin{bmatrix} 1&&0\\0&&1\end{bmatrix}$, $x\mapsto\begin{bmatrix} 0&&1\\0&&0\end{bmatrix}$.
   \\$H_4$ is a quantum group with an $R$-matrix $R=1\otimes 1-2\tfrac{1-g}{2}\otimes\tfrac{1-g}{2}+x\otimes x+2x\tfrac{1-g}{2}\otimes x\tfrac{1-g}{2}-2x\otimes x\tfrac{1-g}{2}$.
\end{example}

\begin{example}
 In "Finite Ring Groups", Kac and Paljutkin defined an $8$-dimensional non-commutative, non-cocommutative Hopf Algebra
 \[
   H_8=\langle x,y,z\;\vert\; x^2=y^2=1,xy=yx,xz=zy,yz=zx,z^2=\tfrac{1}{2}(1+x+y-xy)\rangle 
 \]
 with operations
 \begin{align*}
   \triangle(x)&=x\otimes x, & \triangle(y)&=y\otimes y, & \triangle(z)&=\tfrac{1}{2}(1\otimes 1\otimes +1\otimes x+y\otimes 1-y\otimes x)(z\otimes z)\\
   \varepsilon(x)&=1, & \varepsilon(y)&=1, & \varepsilon(z)&=1\\
   S(x)&=x, & S(y)&=y, & S(z)&=z
 \end{align*}
$H_8$ has as $H_8$-module algebras, $\mathbb{C}_q[v_1,v_2]$ with $q^2=-1$, $\mathbb{C}_Q[v_1,v_2,v_3,v_4]$ with $q_{12}=q_{34}^{-1}, \\q_{13}=q_{24}^{-1}, q_{14}^2=1$ and $q_{23}^2=-1$, and $\mathbb{C}_{-1}[v_1,v_2]$.
$H_8$ is a quantum group with $6$ non-isomorphic quasitriangular structures.
\end{example}

\begin{example}
Described by Kulish and Reshetikhin in "Quantum linear problem for the sine-Gordon equation and higher representations", 
\[
\mathcal{U}_q(\mathfrak{sl_2})=\langle E,F,K,K^{-1}\;\vert\; EF-FE=(q-q^{-1})^{-1}(K-K^{-1}), KEK^{-1}=q^2E, KFK^{-1}=q^{-2}F, \]\[KK^{-1}=1=K^{-1}K\rangle
\]
with operations
\begin{align*}
\triangle(E)&=E\otimes 1+K\otimes E, & \triangle(F)&=F\otimes K^{-1}+1\otimes F, & \triangle(K)&=K\otimes K, & \triangle(K^{-1})&=K^{-1}\otimes K^{-1},\\
\varepsilon(E)&=0, & \varepsilon(F)&=0, & \varepsilon(K)&=1, & \varepsilon(K^{-1})&=1,\\
S(E)&=-K^{-1}E, & S(F)&=-FK, & S(K)&=K^{-1}, & S(K^{-1})&=K
\end{align*}
$\mathbb{C}_q[v_1,v_2]$ is a $\mathcal{U}_q(\mathfrak{sl_2})$-module algebra where $q^2\neq 1$ via $E\mapsto \begin{bmatrix}0&1\\0&0\end{bmatrix}, F\mapsto \begin{bmatrix}0&0\\1&0\end{bmatrix}, \\K\mapsto \begin{bmatrix}q&0\\0&q^{-1}\end{bmatrix}, K^{-1}\mapsto\begin{bmatrix} q^{-1}&0\\0&1\end{bmatrix}$.
\\The $R$-matrix of $\mathcal{U}_q(\mathfrak{sl_2})$ is in a completion of $\mathcal{U}_q(\mathfrak{sl_2})\otimes \mathcal{U}_q(\mathfrak{sl_2})$.
\end{example}

\section{Smash Product Algebras}

\begin{definition}
Given a Hopf algebra $H$ and a left Hopf-module algebra $A$, the smash product algebra $A\# H$ is the algebra where $A\#H=A\otimes H$ as a $\mathbb{C}$-vector space and has product
\[
(a\otimes h)(b\otimes k)=\sum a(h\1\cdot b)\otimes h\2 k.
\]
\end{definition}

\begin{definition}
    Let $H$ be a Hopf algebra, $G(H)=\{h\in H\;\vert\; \triangle(h)=h\otimes h\}$ is called the collection of group-like elements and $P(H)=\{h\in H\;\vert\; \triangle(h)=h\otimes 1+1\otimes h\}$ is called the collection of primitive elements.
\end{definition}

\begin{theorem}[Cartier-Kostant-Milnor-Moore]
Let $H$ be a cocommutative Hopf algebra over $\mathbb{C}$, then as Hopf algebras
\[
H\cong \mathcal{U}(P(H))\# \mathbb{C}G(H).    
\]
\end{theorem}

\end{document}