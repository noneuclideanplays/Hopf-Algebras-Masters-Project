\documentclass[12pt,a4paper]{article}
\usepackage[english]{babel}
\usepackage[utf8x]{inputenc}
\usepackage[T1]{fontenc}
\usepackage{listings}
\usepackage{amsfonts}
\usepackage{amssymb}
\usepackage{amsmath,amsthm}
\usepackage{mathtools}
\usepackage{graphicx}
\usepackage{geometry}
\usepackage{layout}
\usepackage{tikz}
\usepackage{bbm}
\usepackage{tikz-cd}
\usepackage{cite}
\usetikzlibrary{positioning}
\usepackage{latexsym,amsxtra,amscd,ifthen, amsmath,color,url}
\newcommand{\comp}{\overline{\phantom{A}}}
\setlength{\tabcolsep}{12pt}

\newtheorem{theorem}{Theorem}[section]
\newtheorem{lemma}[theorem]{Lemma}
\newtheorem{claim}[theorem]{Claim}
\newtheorem{proposition}[theorem]{Proposition}
\newtheorem{corollary}[theorem]{Corollary}
\newtheorem{fact}[theorem]{Fact}
\newtheorem{example}[theorem]{Example}
\newtheorem{notation}[theorem]{Notation}
\newtheorem{observation}[theorem]{Observation}
\newtheorem{conjecture}[theorem]{Conjecture}
\newtheorem{remark}[theorem]{Remark}
\newtheorem{definition}[theorem]{Definition}
\newcommand\NN{\mathbb{N}} 
\newcommand\RR{\mathbb{R}}
\newcommand\CR{\mathcal{R}}
\newcommand\ZZ{\mathbb{Z}}
\newcommand\AL{\mathcal{A}}
\newcommand\MM{\mathcal{M}}
\newcommand\BB{\mathcal{B}}
\newcommand\BBB{\mathfrak{B}}
\newcommand\CC{\mathcal{C}}
\newcommand\CCC{\mathfrak{C}}
\newcommand\DD{\mathcal{D}}
\newcommand\EE{\mathbb{E}}
\newcommand\FF{\mathcal{F}}
\newcommand\GG{\mathcal{G}}
\newcommand\HH{\mathcal{H}}
\newcommand\JJ{\mathcal{J}}
\newcommand\PP{\mathcal{P}}
\newcommand\QQ{\mathbb{Q}}
\newcommand\CS{\mathcal{S}}
\newcommand\TT{\mathcal{T}}
\newcommand\BT{\mathbb{T}}
\newcommand\OO{\mathcal{O}}
\newcommand\UU{\mathcal{U}}
\newcommand\VV{\mathcal{V}}
\newcommand{\kk}{\Bbbk}
\newcommand\PPT{\mathcal{P}^{*}}
\newcommand\CST{\Sigma^{*}}
\newcommand\RX{\mathsf{X}}
\newcommand\1{_{(1)}}
\newcommand\2{_{(2)}}

%\voffset=-1truein		% LaTeX has too much space at page top
%\addtolength{\textheight}{0.3truein}
%\addtolength{\textheight}{\topmargin}
%\addtolength{\topmargin}{-\topmargin}
%\textwidth  6.0in		% LaTeX article default 360pt=4.98''
%\oddsidemargin 0pt	% \oddsidemargin  .35in   % default is 21.0 pt
%\evensidemargin 0pt	% \evensidemargin .35in   % default is 59.0 pt
%\marginparwidth=0pt
% decrease margins on sides and top/bottom
\geometry{
bottom=25mm,
top=35mm,
left=20mm,
right=20mm
}
\voffset=-0.5truein
\title{Hopf Algebras Acting on Quantum Planes}
\date{}
\author{Brandon Mather}
\begin{document}%\layout
	%\MakeScribeTop

\maketitle

\section{Hopf Algebra Definitions}
\begin{definition}
A Hopf Algebra, $\left(H,\bigtriangledown,\eta,\triangle,\varepsilon,S\right)$, is a bialgebra $H$ over a field $\kk$ with an antipode $S:H\to H$ where the bialgebra has product $\bigtriangledown:H\otimes H\to H$, unit $\eta:\kk\to H$, coproduct $\triangle:H\to H\otimes H$, counit $\varepsilon:H\to\kk$ such that the following diagrams commute
% https://tikzcd.yichuanshen.de/#N4Igdg9gJgpgziAXAbVABwnAlgFyxMJZABgBoBGAXVJADcBDAGwFcYkQBBAHS4jwFt4AAg4gAvqXSZc+QijLFqdJq3bdeA4er5ZBcEeMkgM2PASLlSimgxZtEnHjr0GJU07IsUlt1Q9FiSjBQAObwRKAAZgBOEPxIliA4EEhkynbsWFBOmvo8ONFY9GAhjMEQAO6EbiAxcQk0yUgATDSM9ABGMIwACtJmciBlkTggNir2IPmFxaXlVTm6wlmGUbHxiGlNiADM4xkO00UlZVCV1UZ1G61JKbv7flNcBcdzZ1XilGJAA
\[
\text{Associativity:}
\begin{tikzcd}
H\otimes H\otimes H \arrow[d, "id\otimes \bigtriangledown"] \arrow[r, "\bigtriangledown\otimes id"] & H\otimes H \arrow[d, "\bigtriangledown"] \\
H\otimes H \arrow[r, "\bigtriangledown"]                                                         & H                                    
\end{tikzcd}
\text{   \;\;\;\;Unit:}
% https://tikzcd.yichuanshen.de/#N4Igdg9gJgpgziAXAbVABwnAlgFyxMJZARgBoAGAXVJADcBDAGwFcYkQBBAHS4jwFt4AAg4gAvqXSZc+QinKli1Ok1bseAaw08+WQXBHjJIDNjwEiZAEzKGLNok5GpZ2USuLbqh5x0DhmhriyjBQAObwRKAAZgBOEPxIZCA4EEgKKvbqXDA49H56wlhQziBxCUgAzDSp6TR2ao7FBfo8ufQgNIz0AEYwjAAK0uZyILFYYQAWOKXliYjJtYgemY0gALydIN19g8NujuNTMxIx8fPVKWnL9d7sm6dl53VXSCsNPjw44-RgYYyhCAAd0IYkoYiAA
\begin{tikzcd}
                                                           & H\otimes H \arrow[dd, "\bigtriangledown"] &                                                           \\
\kk\otimes H \arrow[ru, "\eta\otimes id"] \arrow[rd, "="'] &                                        & H\otimes \kk \arrow[lu, "id\otimes\eta"'] \arrow[ld, "="] \\
                                                           & H                                      &                                                          
\end{tikzcd}
\]
\[
\text{Coassociativity:} % https://tikzcd.yichuanshen.de/#N4Igdg9gJgpgziAXAbVABwnAlgFyxMJZABgBpiBdUkANwEMAbAVxiRAAkQBfU9TXfIRQBGclVqMWbdgB0ZEPAFt4AAk48+2PASJlh4+s1aIOchVmVw13XiAxbBRUfuqGpJ2fKWrP5y9a5xGCgAc3giUAAzACcIRSQyEBwIJFEJIzY5HGisOjAQhlYNEBi4hOpkpAAmV0ljECycvIKi21L4xBqklMQAZlqMk0bc-MKQajgACyxInATi9tSKnv709waZbJGWs28rLChuCi4gA
\begin{tikzcd}
H \arrow[r, "\triangle"] \arrow[d, "\triangle"] & H\otimes H \arrow[d, "\triangle\otimes id"] \\
H\otimes H \arrow[r, "id\otimes\triangle"]               & H\otimes H\otimes H                        
\end{tikzcd}
\text{\;\;\;\; Counit:}
% https://tikzcd.yichuanshen.de/#N4Igdg9gJgpgziAXAbVABwnAlgFyxMJZABgBoBGAXVJADcBDAGwFcYkQAdDga264jwBbeAAIAEiAC+pdJlz5CKcqWLU6TVuwnTZ2PASIAmCmoYs2iEGP5D4XXlJkgMehUWWHTGi1ZtZhcOJSajBQAObwRKAAZgBOEIJIAMw0OBBIZOrm7FwMsTBo2IwEfgEiWFCOMfGJiMogaRk0ZpqWALwgNIz0AEYwjAAKcvqKILFYYQAWOFUgcQlI9Y2IxlmtIB06czXJqekrzd7sFaV2HHkFRQZdvf1DrgaW41MzW-O1S-spaz5cOOP0MBhRhsSSUSRAA
\begin{tikzcd}
             & H \arrow[ld, "="'] \arrow[rd, "="] \arrow[dd, "\triangle"]                         &             \\
\kk\otimes H &                                                                                    & H\otimes\kk \\
             & H\otimes H \arrow[lu, "\varepsilon\otimes id"] \arrow[ru, "id\otimes\varepsilon"'] &            
\end{tikzcd}
\]
\[
\text{Coproduct compatibility:}
% https://tikzcd.yichuanshen.de/#N4Igdg9gJgpgziAXAbVABwnAlgFyxMJZABgBpiBdUkANwEMAbAVxiRAAkAdTiPAW3gACdiAC+pdJlz5CKAIzkqtRizYjxk7HgJEATIur1mrRB268sAuMLESQGLTKJk5So6tNce-IV4tXhcx9rdTsHaR0UfVdDFRMzb0tfIKSQlID1JRgoAHN4IlAAMwAnCD4kMhAcCCQAZljjNm4cYqw6MByGGHT4Ztb2ztYNEBKyuupqpAAWBo8QLCgeuGa6JiXBBdsi0vLESsnEBWVG0z62jq6oCAB3QmHR3aOD-WO5s4GurZGd6YmaxBe7ni7wu2RuYCWIMGV1uYgooiAA
\begin{tikzcd}
H\otimes H \arrow[d, "\triangle\otimes\triangle"] \arrow[r, "\bigtriangledown"] & H \arrow[r, "\triangle"] & H\otimes H                                                                  \\
H\otimes H\otimes H\otimes H \arrow[rr, "id\otimes\tau\otimes id"]           &                          & H\otimes H\otimes H\otimes H \arrow[u, "\bigtriangledown\otimes\bigtriangledown"]
\end{tikzcd}
\]
\[
% https://tikzcd.yichuanshen.de/#N4Igdg9gJgpgziAXAbVABwnAlgFyxMJZABgBpiBdUkANwEMAbAVxiRAB12BrLkAX1LpMufIRQBGclVqMWbABL9BIDNjwEik8dPrNWiEPM4Q8AW3gACRX2kwoAc3hFQAMwBOEU0jIgcEJABM1LpyBpwwOHTGZvDhkUquHl6Ikr7+iEEyemycOG5YdGD2DKwCiZ7e1H5IqSH6HOwRdPwUfEA
\text{Unit compatibility:}\begin{tikzcd}
\kk \arrow[rd, "\eta\otimes\eta"] \arrow[r, "\eta"] & H \arrow[d, "\triangle"] \\
                                                    & H\otimes H              
\end{tikzcd}
\text{\;\;\;Counit compatibility:}
% https://tikzcd.yichuanshen.de/#N4Igdg9gJgpgziAXAbVABwnAlgFyxMJZABgBpiBdUkANwEMAbAVxiRAAkQBfU9TXfIRQBGclVqMWbADrSA1nO68QGbHgJEyw8fWatEHWRDwBbeAAJOXcTCgBzeEVAAzAE4QTSAEzUcEJGQSejLSOK5YdGB2DLYQAO6EPC7unog+IH5IokFSBrL0rjBo2AwERqbw+XSFxVilhNQMdABGMAwACvzqQiDhdgAWOErJHgG+-ojZurkgVTUlGtZcQA
\begin{tikzcd}
H \arrow[r, "\varepsilon"]                                                         & \kk \\
H\otimes H \arrow[u, "\bigtriangledown"] \arrow[ru, "\varepsilon\otimes\varepsilon"'] &    
\end{tikzcd}
\]
\[
\text{Antipode:}
% https://tikzcd.yichuanshen.de/#N4Igdg9gJgpgziAXAbVABwnAlgFyxMJZABgBoBGAXVJADcBDAGwFcYkQAJEAX1PU1z5CKchWp0mrdgB1pAazk8+IDNjwEiAJjE0GLNok5L+aoUTKbxeqYY6yIeALbwABF14nBGlNsu7JBpz2Tq7uyqpewiSkxFYB7HbSDljOcG7GKgLqUdqx-voJwSmhPOIwUADm8ESgAGYAThCOSGQgOBBIAKz5NiCyOPVY9GAVjGweIA1NXTTtSABsPYFYUEWpLgDKGVPNiIttHYjaEgWG-YPDo+UQAO6EEzsts4cAzEsy0gNDI2MgNIz0ABGMEYAAUsmZDIMKgALHDbRq7N4HJAAFnehg2a1cKwR00Q6JRRwxfU+Fx+1zueN2rTmiFEJ16sgY9RgaGwjA0D0RSAZdOO1kCshgOHopW4QA
\begin{tikzcd}
H\otimes H \arrow[rr, "id\otimes S"]                                      &                       & H\otimes H \arrow[d, "\bigtriangledown"] \\
H \arrow[u, "\triangle"] \arrow[d, "\triangle"'] \arrow[r, "\varepsilon"] & \kk \arrow[r, "\eta"] & H                                     \\
H\otimes H \arrow[rr, "S\otimes id"]                                      &                       & H\otimes H \arrow[u, "\bigtriangledown"]
\end{tikzcd}
\]\\
\end{definition}

For the sake of brevity, we write in general that $\triangle(h)=\sum h\1\otimes h\2$, this is called Sweedler notation.\\

One can note that these diagrams are self-dual, changing the directions of the morphisms gives another diagram. 
Then an immediate question is when is the dual of a Hopf algebra again a Hopf algebra?

\begin{definition}
    Let $V$ be a $\kk$ vector space and $V^*$ its corresponding dual, then they determine a non-degenerate bilinear form $\langle,\rangle:V^*\otimes V\to\kk$ by $\langle\phi,v\rangle=\phi(v)$.
\end{definition}

\begin{definition}
    If $V$ and $W$ are $\kk$ vector spaces and $f:V\to W$ is $\kk$-linear, then the transpose of $f$ is $f^*:W^*\to V^*$ given by 
    \[
    f^*(\phi)(v)=f(\phi(v)).
    \]
\end{definition}

\begin{definition}
    Let $(C,\triangle,\varepsilon)$ be a coalgebra, then $C^*$ is an algebra with multiplication \\$\triangle^*:C^*\otimes C^*\to C^*$ and unit $\varepsilon^*:\kk\to C^*$.
\end{definition}

Note that $\triangle^*$, by definition $1.3$, maps from $(C\otimes C)^*$, but we can restrict the map to the domain $C^*\otimes C^*$ to meet the criteria of being a product.\\

In a similar vein, if we start with an algebra $(A,\bigtriangledown,\eta)$, then the transpose of the product $\bigtriangledown$ is $\bigtriangledown^*:A^*\to (A\otimes A)^*$.
But unless $A$ is finite dimensional, we cannot know that $\bigtriangledown^*(A^*)\subseteq A^*\otimes A^*$, which is required for $\bigtriangledown^*$ to be a coproduct.
This is exactly the requirement for $A^*$ to be a coalgebra.
This motivates the following definition.

\begin{definition}
The finite dual of an algebra $H$ is $H^\circ=\{f\in H^*\;\vert\; f(I)=0$ for some ideal $I$ of A where $\dim H/I<\infty\}$. 
\end{definition}

If $H$ is finite-dimensional, then $H^\circ$ is exactly $H^*$.

\begin{proposition}
If $A$ is an algebra, then $A^\circ$ is a coalgebra with\\ coproduct $\bigtriangledown^*:A^\circ\to(A\otimes A)^\circ=A^\circ\otimes A^\circ$ and counit $\eta^*:A^\circ\to\kk$.
\end{proposition}

\begin{proposition}
As proved in \cite{Maj}, if $H$ is a Hopf algebra, $H^\circ$ is also a Hopf algebra with product, unit, coproduct, counit and antipode $\triangle^*$, $\varepsilon^*$, $\bigtriangledown^*$ $\eta^*$, $S^*$ respectively.
Explicitly, $\forall \phi,\psi\in H^\circ$ and all $h,g\in H$,
\[
\langle\bigtriangledown^*(\phi\psi),h\rangle=\langle\phi\otimes\psi,\triangle(h)\rangle , \;\; \langle1,h\rangle=\varepsilon(h), \;\; \langle\triangle^*(\phi),h\otimes g\rangle =\langle\phi, \bigtriangledown(hg)\rangle,\;\; \varepsilon^*(\phi)=\langle\phi,1\rangle ,
\]\[
\langle S^*\phi h\rangle =\langle \phi,Sh\rangle .
\]
\end{proposition}

\begin{example}
    Let $G$ be any group and denote $\kk G=\left\{\sum_{i=0}^\infty a_i g_i\;\vert\; a_i\in\kk, g_i\in G, n\in\NN\right\}$
    The $\kk G$ is a Hopf algebra called the group algebra of $G$.
    It has the product
    \[
    \left(\sum_{i=-0}^n a_i g_i\right)\left(\sum_{j=0}^m b_j g_j\right)=\sum_{i=o}^n\sum_{j=0}^m a_i b_j (g_i g_j)
    \]
    where $a_i\cdot b_j$ is the product in $\kk$ and $g_i\cdot g_j$ is the product in the group.
    The unit is $1_\kk 1_G$ where $1_\kk$ is the unit of $\kk$ and $1_G$ is the identity element of the group.
    The coproduct is defined by $\triangle(g)=g\otimes g$ extended linearly to all of $\kk G$, and the counit is $\varepsilon(g)=1_\kk$, again extended linearly.
    Finally, the antipode is $S(g)=g^{-1}$.
\end{example}

Note that group algebras are always cocommutative, in other words $\bigtriangledown(h)=\tau\circ\bigtriangledown(h)$ for all $h\in \kk G$, where $\tau(a\otimes b)=b\otimes a$, and are commutative if and only if $G$ is abelian.

\begin{definition}
    For a Hopf algebra H, $G(H)=\{g\in H\;\vert\; \triangle{g}=g\otimes g\}$ is called the set of grouplike elements of $H$.
\end{definition}

\begin{example}
    Let $\mathfrak{g}$ be a Lie algebra and $U(\mathfrak{g})$ the corresponding Universal Enveloping algebra.
    Then $U(\mathfrak{g})$ is naturally an algebra, but also has a Hopf algebra structure.
    The coproduct is given by $\triangle{x}=x\otimes 1+1\otimes x$, $\varepsilon(x)=0$ and $S(x)=-x$.
\end{example}

\begin{definition}
    For a Hopf algebra H, $P(H)=\{x\in H\;\vert\; \triangle{x}=x\otimes 1+1\otimes x\}$ is called the set of primitive elements of $H$.
    Generally, one can define the skew-primitive elements as $P_{a,b}=\{x\in H\;\vert\; \triangle{x}=x\otimes a+b\otimes x\}$.
\end{definition}

\begin{example}
    In his seminal book, \cite{Sw}, Sweedler defined a 4-dimensional, non-commutative, non-cocommutative Hopf algebra
    \[
    H_4=\langle g,x\;\vert\; g^2=1, x^2=0,gx=-xg\rangle
    \]
    with operations
    \[
    \triangle{g}=g\otimes g,\;\; \triangle{x}=x\otimes 1+g\otimes x,\;\;\varepsilon(g)=1,\;\;\varepsilon(x)=0,
    \]
    \[
    S(g)=g^{-1}\;\;S(x)=-xg^{-1}.
    \]
\end{example}

We will see in section 3 a generalization of this to Taft algebras, which were introduced by Taft in \cite{T}.

\begin{definition}
If H is a Hopf algebra, then $H^\text{op}$ is a Hopf algebra with the same structure except the opposite multiplication, $\bigtriangledown^\text{op}(hg)=\bigtriangledown(gh)$.
As well, $H^\text{cop}$ is a Hopf algebra with the same structure as $H$ but with the opposite coproduct, $\triangle^\text{cop}(h)=\sum h\2\otimes h\1$.
\end{definition}

\begin{definition} A left action of a Hopf algebra on a vector space $V$ is a tuple $(\alpha, V)$ so that
$\alpha:H\otimes V\to V$ is a map satisfying the diagrams
\[
% https://tikzcd.yichuanshen.de/#N4Igdg9gJgpgziAXAbVABwnAlgFyxMJZABgBpiBdUkANwEMAbAVxiRAAkAdTiPAW3gACLj35CAaiAC+pdJlz5CKMgEYqtRizYjeWAXEGSZc7HgJEV5dfWatEHbrv2HpskBlOKLpNdRtb7I3UYKABzeCJQADMAJwg+JDIQHAgkACY-TTsQbhwYrDowUIYQiAB3MEcxAywoV2i4hMQklKRLDVs2Wqq9IW5GNAALOnqQWPi26lbEAGZMzvt+hiGR4zHG9KnU2fmAnM4B4ekKKSA
\begin{tikzcd}
H\otimes H\otimes V \arrow[r, "\bigtriangledown\otimes id"] \arrow[d, "id\otimes \alpha"] & H\otimes V \arrow[d, "\alpha"] \\
H\otimes V \arrow[r, "\alpha"]                                                         & V                             
\end{tikzcd}
% https://tikzcd.yichuanshen.de/#N4Igdg9gJgpgziAXAbVABwnAlgFyxMJZABgBpiBdUkANwEMAbAVxiRAB12BrLziPALbwABADUQAX1LpMufIRQBGclVqMWbABJ9BI8VJnY8BIssWr6zVohD7VMKAHN4RUADMAThAFIyIHBBIympWbJwwOHQ6WEJwwlhQktIgnt6+1AFIAEzUlho2ALwg1Ax0AEYwDAAKssYKIB5YjgAWOEnuXj6IwZmIOSH5HOyMaM10khQSQA
\begin{tikzcd}
\kk\otimes V \arrow[r, "\eta\otimes id"] \arrow[rd, "="'] & H\otimes V \arrow[d, "\alpha"] \\
                                                          & V                             
\end{tikzcd}
\]
In this case, $V$ is called a left Hopf-module.
\end{definition}

For the rest of this paper we suppress the action $\alpha$ and instead write $\alpha(h\otimes v)=\prescript{h}{}v$.

\begin{definition}
A right coaction, $\rho$, of $H$ on $V$ is a tuple $(\rho,V)$ with $\rho:V\to V\otimes H$ so that the following commute\[
% https://tikzcd.yichuanshen.de/#N4Igdg9gJgpgziAXAbVABwnAlgFyxMJZABgBpiBdUkANwEMAbAVxiRADUQBfU9TXfIRRkAjFVqMWbdgB0ZEPAFt4AAgAS3XiAzY8BIiNJjq9Zq0Qc5CrMrjqrS1Rp59dgg+XGmpF2fMd2zuIwUADm8ESgAGYAThCKSGQgOBBIAMwmkuYgcjEAFqkuILHxidQpSIYSZmy5BSDUDHQARjAMAAr8ekIgMViheTia0XEJiFUViABMmTUWdRAONqpYUA0gTa0dXe4WfQNDRSVjGcmp07M+IKtLtipyOH10YKEMrFwUXEA
\begin{tikzcd}
V \arrow[r, "\rho"] \arrow[d, "\rho"']  & V\otimes H \arrow[d, "id\otimes \triangle"] \\
V\otimes H \arrow[r, "\rho\otimes id"'] & V\otimes H\otimes H                        
\end{tikzcd}
% https://tikzcd.yichuanshen.de/#N4Igdg9gJgpgziAXAbVABwnAlgFyxMJZABgBpiBdUkANwEMAbAVxiRADUQBfU9TXfIRQBGclVqMWbdgB0ZEPAFt4AAgAS3XiAzY8BIqOHj6zVog5yFWZXDkBrO93EwoAc3hFQAMwBOERUhkIDgQSKISpmxyPgAWoTzefgGI4SFIAEzUJlLmWFCWSvBy9D4waNgM+gkgvv6B1GmImRE5IAC8INQMdABGMAwACvx6QiA+WK4xOE5cQA
\begin{tikzcd}
V \arrow[r, "\rho"] \arrow[rd, "="'] & V\otimes H \arrow[d, "id\otimes\varepsilon"] \\
                                     & V\otimes\kk                                 
\end{tikzcd}
\]
In this case, $V$ is called a right Hopf-comodule.\\
\end{definition}
\begin{definition}
When an algebra $A$ is a left Hopf-module with action $\alpha$, we call it a left Hopf-module algebra if it also satisfies the diagrams
\[
% https://tikzcd.yichuanshen.de/#N4Igdg9gJgpgziAXAbVABwnAlgFyxMJZABgBpiBdUkANwEMAbAVxiRAAkAdTiPAW3gACAILdeWAXBEgAvqXSZc+QigCM5KrUYs2XHvyHDZ8kBmx4CRAEwbq9Zq0QgjcheeVEAzLa0O2o-QlDYzclSxQyVU17HSc9cUlBeIMpAITg11NFCxVkbyi7bUcOMRSRUqCpZMrpGU0YKABzeCJQADMAJwg+JDIQHAgkdV9YkG4cDqw6MEaGBogAd0JMzu6h6gGkGxHi7kY0AAs6EJBVnsRvfsHEbZjdzgmpmbmoRcJqBjoAIxgGAAVsh4nJNGgccCczr0NtcACyFPxOcaTaazGAVRJYKDooSYkAfb6-AHucIgEFgiFdc5wq5IACs8NGmOxcHGdCYzMEuJWlLp0KQlzubD2DEOdGZwtFeJAnx+-0BJLJ4LqMiAA
\begin{tikzcd}
H\otimes A\otimes A \arrow[r, "\bigtriangledown"] \arrow[d, "\triangle\otimes id\otimes id"'] & H\otimes A \arrow[r, "\alpha"] & A & A\otimes A \arrow[l, "\bigtriangledown"']                         \\
H\otimes H\otimes A\otimes A \arrow[rrr, "id\otimes\tau\otimes id"]                        &                                &   & H\otimes A\otimes H\otimes A \arrow[u, "\alpha\otimes\alpha"']
\end{tikzcd}
% https://tikzcd.yichuanshen.de/#N4Igdg9gJgpgziAXAbVABwnAlgFyxMJZABgBpiBdUkANwEMAbAVxiRAAkQBfU9TXfIRQBGclVqMWbADrSA1nO68QGbHgJEATGOr1mrRCACCSvmsFEyw8XqmH2siHgC28AAQmu4mFADm8IlAAMwAnCGckMhAcCCRRCX0ZaXoQmDRsBg0eYLCIxHiYpG0EuxBZGBw6UxBQ8MjqQsQAZl1JAzLpCqrsmtykFujYxGLbdtlGNAALboouIA
\begin{tikzcd}
H \arrow[r, "\varepsilon"] \arrow[d, "\eta"] & \kk \arrow[r, "\eta"] & A \\
H\otimes A \arrow[rru, "\alpha"]             &                       &  
\end{tikzcd}
\]
\end{definition}

Note that a Hopf algebra acts if and only if its dual coacts. (Write why here)

\begin{definition}
And when a coalgebra $A$ is a an Hopf-module with action $\alpha$, we call it an Hopf-module coalgebra if it also satisfies the diagrams
\[
% https://tikzcd.yichuanshen.de/#N4Igdg9gJgpgziAXAbVABwnAlgFyxMJZABgBpiBdUkANwEMAbAVxiRAAkAdTiPAW3gACAIIgAvqXSZc+QijIBGKrUYs2XHvyEbeWAXBHdd+keMkgM2PASILyy+s1aIQoiVKuyiAJnvVHai7CRloGbuaWMjYovkr+qs4cIXpCwZopBjqhpmLKMFAA5vBEoABmAE4QfEhkIDgQSL4qTmzcjGgAFnRmZZXViE31SADM8S0u3DjlWHRgBQys7iAVVTXUQ4h2zYEgk9Oz8zDJJlhQx0KnINQMdABGMAwACtLWciDTBR04Pct9SFsbAAsYx2p3OcEmdCY4MElyWK36wLqDUQo22iTaDE6dHBmOxVxAN3uTxeXhcHy+4goYiAA
\begin{tikzcd}
H\otimes A \arrow[r, "\alpha"] \arrow[d, "\triangle\otimes id\otimes id"'] & A \arrow[r, "\triangle"] & A\otimes A                                                     \\
H\otimes H\otimes A\otimes A \arrow[rr, "id\otimes\tau\otimes id"]         &                          & H\otimes A\otimes H\otimes A \arrow[u, "\alpha\otimes\alpha"']
\end{tikzcd}
% https://tikzcd.yichuanshen.de/#N4Igdg9gJgpgziAXAbVABwnAlgFyxMJZABgBpiBdUkANwEMAbAVxiRAEEQBfU9TXfIRQBGclVqMWbADrSA1nO68QGbHgJEyw8fWatEIABKyIeALbwABJy7iYUAObwioAGYAnCGaRkQOCEiiEnoy0vTuMGjYDBo8bp7eiABM1P4+1LpSBrKMaAAWdErxXkgpfgGIQZn6ILLhkdEEJubwdXQRUVgxhNQMdABGMAwACvzqQiDuWA55ONwUXEA
\begin{tikzcd}
A \arrow[r, "\varepsilon"]                                                  & \kk \\
H\otimes A \arrow[u, "\alpha"] \arrow[ru, "\varepsilon\otimes\varepsilon"'] &    
\end{tikzcd}
\]
\end{definition}

Similar diagrams give the conditions for Hopf-comodule algebras and Hopf-comodule coalgebras.\\


\begin{definition}
A useful and prevalent construction on Hopf algebras is given a Hopf algebra $H$ and a left Hopf-module algebra $A$, the smash product algebra $A\# H$ is the algebra where $A\#H=A\otimes H$ as a $\kk$-vector space and has product
\[
(a\otimes h)(b\otimes k)=\sum a\;\prescript{h\1}{}b\otimes h\2 k.
\]
\end{definition}

\begin{definition}
    A Hopf algebra is called pointed if all of its left (right) comodules are \\$1$-dimensional.
\end{definition}

\begin{definition}
    If $I\subseteq H$ and for any $h\in H$, $hI\subseteq I$, then $I$ is called an ideal of $H$.
    If $\triangle(I)\subseteq I\otimes H+H\otimes I$, then $I$ is called a coideal of $H$.
    If $I$ is both an ideal and a coideal, it is called a biideal.
    Finally, if $I$ is a biideal and $S(I)\subseteq I$, then $I$ is called a Hopf ideal of H.
\end{definition}

\begin{lemma}
    If $I$ is a Hopf ideal of $H$, then the quotient $H/I$ is a Hopf algebra.
\end{lemma}

\begin{definition}
    If $H$ acts on an algebra $A$ and there is a Hopf ideal $I$ so that $I\cdot A=0$, then we say the action of $H$ factors through the quotient $H/I$.
    In particular, if $H/I$ is isomorphic to a group algebra, we say the action of $H$ factors through a group action.
\end{definition}

\section{Big Questions}
Juan Cuadra, Pavel Etingof and Chelsea Walton have been classifying algebras with Hopf actions for which the action factors through a group action\cite{EW}.
For example, they have shown that any action by a semi-simple, finite dimensional Hopf algebra on an integral domain always has this property.
As well, they have shown that any action by a finite dimensional Hopf algebra on a Weyl algebra also has this property.
This is a component of a larger search for algebras on which Hopf algebras act.\\\\
Andruskiewitsch and Schneider have been classifying pointed Hopf algebras\cite{AS}.\\\\
Kenneth Chan, Ellen Kirkman, Jim Kuzmanovich, Chelsea Walton, and James Zhang have a series of works on  Hopf algebras acting on AS-regular algebras \cite{CKWZ} \cite{CKWZ2}\cite{CKWZ3}\cite{KKZ}.
They have posed the question of when are the coinvariant subrings from these actions Artin-Schelter Gorenstein?\\\\
Miriam Cohen and Davida Fishman extended work by Fisher and Montogomery \cite{FM} and Cohen and Montogomery \cite{CM} to determine when $A\#H$ is semiprime for $A$ and algebra and $H$ a Hopf algebra.
Specifically, if $H$ is semi-simple and finite-dimensional and $A$ is semiprime, they ask is $A\#H$ is semiprime?\cite{CF}\\\\
Chelsea Walton and Sarah Witherspoon have been working towards PBW deformation conditions on $B\# H$ where $B$ is a Koszul algebra and $H$ a Hopf algebra.
In \cite{WW} they are able to provide these conditions when the antipode of $H$ is bijective, $B$ is connected as an $H$-module algebra, and the action of $H$ preserves the grading on $B$.
In the same paper they pose the question if $H=U_q(\mathfrak{sl}_2)$, are there nontrivial PBW deformations of $B\#H$?



\section{Taft Algebras}
A Taft algebra, as defined in \cite{CKWZ}, is a Hopf algebra $T_{n,m}=\left<g,x\;\vert\; g^n=1,x^n=0,gx=\zeta xg\right>$ where $\zeta$ is a primitive $n$-th root of unity. 
$T_{n,m}$ has the maps
\[
\triangle(g)=g\otimes g,\;\;\;\; \triangle(x)=1\otimes x+x\otimes g
\]\[
\varepsilon(g) =1,\;\;\;\; \varepsilon(x)=0,\;\;\;\;
S(g)=g^{-1}, \;\;\;\; S(x)=-xg^{-1}.
\]

A small example is the lowest dimension non-commutative, non-cocommutative Hopf algebra, the 4-dimensional Sweedler algebra.
This is given by $H_4=\langle g,x\;\vert\; g^2=1, x^=0, xg=-gx\rangle$ with operations
\[
\triangle(g)=g\otimes g,\;\;\;\; \triangle(x)=1\otimes x+x\otimes g    
\]
\[
\varepsilon(g)=1,\;\;\;\; \varepsilon(x)=0,\;\;\;\; S(g)=g, \;\;\;\; S(x)=-xg.    
\]
This algebra in particular acts on the AS-regular algebra $\kk_{-1}[u,v]=\kk[u,v]/(uv+vu)$ by
\[
g\cdot u=u,\;\;\;\; g\cdot v=-v,\;\;\;\; x\cdot u=0,\;\;\;\; x\cdot v=v.    
\]

Let $q$ be a primitive root of unity where $|q^2| =m>1$, let $\alpha\in \kk$ and $n\in \ZZ^+$ where $|q| \big\vert n$, then a generalized Taft algebra is a Hopf algebra 

\[T_{q,\alpha,n}=\left<g,g^{-1}x\;\vert\; gg^{-1}=g^{-1}g=1, xg=q gx, g^n=1, x^m=\alpha(g^m-g^{-m})\right>
\]
with maps
\[
\triangle(g)=g\otimes g,\;\;\;\; \triangle(x)=x\otimes g^{-1}+g\otimes x,
\]
\[
\varepsilon(g)=1,\;\;\;\;\varepsilon(x)=0,\;\;\;\; S(g)=g^{-1},\;\;\;\; S(x)=-q x.
\]
If $q^m\neq 1$, then $\alpha=0$, and if $q^m=1$, $\alpha\in\{0,1\}$.\\

Another definition for a generalized Taft algebra given in \cite{AS} is to take a Yetter-Drinfeld module $V$ of $D_2$-type over the group algebra $k\ZZ/n\ZZ$. 
Then if $\BB(V)$ is the Nichols algebra of $V$, the bosonization $(\BB(V)\#k\ZZ/n\ZZ)^\text{cop}$ is a generalized Taft algebra.\\

Let $q$ be a primitive root of unity with $q^2\neq 1$, then we can define the quantum polynomial ring $\kk_q[u,v]=\kk[u,v]/\langle uv-qvu\rangle$.
The subring $U=\kk u\oplus \kk v$ is then a left $T_{q,\alpha,n}$-module.
If as a $T_{q,\alpha,n}$-module $U$ is not semi-simple, then in \cite{CKWZ} it is proved that $T_{q,\alpha,n}$ coacts on $\kk_q[u,v]$.
This coaction is given by 
\[
\rho(u)=u\otimes g\;\;\;
\rho(v)=v\otimes g^{-1}+u\otimes x.
\]
The coaction induces an action by the finite dual $\left(T_{q,\alpha,n}\right)^\circ$ on $\kk_q[u,v]$.

\section{8-dimensional Hopf Algebra}
In \cite{KP}, Kac and Paljutkin define an 8-dimensional, non-commutative, non-cocommutative, semi-simple Hopf algebra
\[
H_8=\langle x,y,z\;\vert\; x^2=y^2=1, xy=yx, zx=yz, zy=xz, z^2=\frac{1}{2}(1+x+y-xy)\rangle
\]
with operations
\[
\triangle(x)=x\otimes x,\;\; \triangle(y)=y\otimes y,\;\; \triangle(z)=\frac{1}{2}(1\otimes 1+1\otimes x+y\otimes 1-y\otimes x)(z\otimes z),
\]
\[
\varepsilon(x)=\varepsilon(y)=\varepsilon(z)=1,\;\; S(x)=x^{-1},\;\; S(y)=y^{-1},\;\; S(z)=z.
\]
Then in \cite{KKJ}, three representations of $H_8$ acting on quantum polynomial rings are given.
For the ring $\kk_q[x_1,x_2]$ where $q^2=-1$, the first representation is given by
\[
x\mapsto \begin{bmatrix}-1 & 0\\0 & 1\end{bmatrix} \;\; y\mapsto \begin{bmatrix} 1 & 0\\ 0 & -1\end{bmatrix} \;\; z\mapsto\begin{bmatrix} 0 & 1\\ 1 & 0\end{bmatrix}.
\]
For the ring $\kk_Q[x_1,x_2,x_3,x_4]$ where $Q=(q_{ij})$, $x_jx_i=q_{ij}x_ix_j$ and $q_{12}=q^{-1}_{34}$, $q_{13}=q^{-1}_{24}$, $q^2_{14}=1$, $q^2_{23}=-1$, we also have the representation 
\[
x\mapsto \begin{bmatrix} 1&0&0&0\\0&-1&0&0\\0&0&1&0\\0&0&0&1\end{bmatrix} \;\; y\mapsto \begin{bmatrix} 1&0&0&0\\0&1&0&0\\0&0&-1&0\\0&0&0&1\end{bmatrix}\;\; z\mapsto\begin{bmatrix}0&0&0&1\\0&0&1&0\\0&1&0&0\\1&0&0&0\end{bmatrix}.
\]
And finally, for the ring $\kk_{-1}[u,v]$, we get the representation
\[
x\mapsto \begin{bmatrix}0&1\\1&0\end{bmatrix}\;\; y\mapsto\begin{bmatrix}0&-1\\-1&0\end{bmatrix}\;\; z\mapsto\begin{bmatrix}1&0\\0&-1\end{bmatrix}.
\]

\[
\lhook
\]

\section{Quantum Enveloping Algebras}



\begin{thebibliography}{99}
\bibitem{AS}Andruskiewitsch, N., and H. -J. Schneider. “Pointed Hopf Algebras,” 2001. https://doi.org/10.48550/ARXIV.MATH/0110136.

\bibitem{CKWZ}Chan, Kenneth, Ellen Kirkman, Chelsea Walton, and James Zhang. “Quantum Binary Polyhedral Groups and Their Actions on Quantum Planes.” arXiv, July 2, 2014. http://arxiv.org/abs/1303.7203.

\bibitem{CKWZ2}Chan, Kenneth, Ellen Kirkman, Chelsea Walton, and James Zhang. “McKay Correspondence for Semisimple Hopf Actions on Regular Graded Algebras, II.” arXiv, October 2, 2017. http://arxiv.org/abs/1610.01220.

\bibitem{CKWZ3}Chan, Kenneth, Ellen Kirkman, Chelsea Walton, and James Zhang. “McKay Correspondence for Semisimple Hopf Actions on Regular Graded Algebras, I.” arXiv, May 12, 2018. http://arxiv.org/abs/1607.06977.

\bibitem{CF}Cohen, Miriam, and Davida Fishman. “Hopf Algebra Actions.” Journal of Algebra 100, no. 2 (May 1986): 363–79. https://doi.org/10.1016/0021-8693(86)90082-7.

\bibitem{CM}Cohen, M, and S Montgomery. “GROUP-GRADED RINGS, SMASH PRODUCTS, AND GROUP ACTIONS,” n.d.

\bibitem{EW}Etingof, Pavel, and Chelsea Walton. “Finite Dimensional Hopf Actions on Deformation Quantizations.” arXiv, July 2, 2016. http://arxiv.org/abs/1602.00532.

\bibitem{FM}Fisher, Joe W, and Susan Montgomery. “Semiprime Skew Group Rings.” Journal of Algebra 52, no. 1 (May 1978): 241–47. https://doi.org/10.1016/0021-8693(78)90272-7.

\bibitem{H}Halbig, Sebastian. “Generalised Taft Algebras and Pairs in Involution.” Communications in Algebra 49, no. 12 (December 2, 2021): 5181–95. https://doi.org/10.1080/00927872.2021.1939043.

\bibitem{KP}Kac, G. I., and V. G. Paljutkin. “Finite Ring Groups.” Trudy Moskov. Mat. Ob\v s\v c. 15 (1966): 224–61.

\bibitem{KKZ}Kirkman, E., J. Kuzmanovich, and J.J. Zhang. “Gorenstein Subrings of Invariants under Hopf Algebra Actions.” Journal of Algebra 322, no. 10 (November 2009): 3640–69. https://doi.org/10.1016/j.jalgebra.2009.08.018.

\bibitem{KKJ}Kirkman, E., J. Kuzmanovich, and J.J. Zhang. “Gorenstein Subrings of Invariants under Hopf Algebra Actions.” Journal of Algebra 322, no. 10 (November 2009): 3640–69. https://doi.org/10.1016/j.jalgebra.2009.08.018.

\bibitem{Maj}Majid, Shahn. Foundations of Quantum Group Theory. Cambridge University Press 1995.

\bibitem{M}Masuoka, A., Semisimple Hopf algebras of dimension 6, 8, Israel J. Math. 92 (1–3) (1995) 361–373

\bibitem{Mon}Montgomery, Susan. Hopf Algebras and Their Actions on Rings. Vol. 82. Providence, R.I.; 4: Published for the Conference Board of the Mathematical Sciences by the American Mathematical Society, 1993.

\bibitem{Sw}Sweedler, Moss E. Hopf Algebras. New York: W. A. Benjamin, 1969.

\bibitem{T}Taft, Earl J. “The Order of the Antipode of Finite-Dimensional Hopf Algebra.” Proceedings of the National Academy of Sciences 68, no. 11 (November 1971): 2631–33. https://doi.org/10.1073/pnas.68.11.2631.

\bibitem{WW}Walton, Chelsea, and Sarah Witherspoon. “Poincaré–Birkhoff–Witt Deformations of Smash Product Algebras from Hopf Actions on Koszul Algebras.” Algebra \& Number Theory 8, no. 7 (October 21, 2014): 1701–31. https://doi.org/10.2140/ant.2014.8.1701.
\end{thebibliography}


\end{document}